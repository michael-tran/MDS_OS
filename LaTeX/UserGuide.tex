\documentclass[10pt]{article}

\usepackage{float}

\usepackage{fancyhdr} % Required for custom headers

\usepackage{lastpage} % Required to determine the last page for the footer

\usepackage{extramarks} % Required for headers and footers

\usepackage{graphicx} % Required to insert images

\usepackage{lipsum} % Used for inserting dummy 'Lorem ipsum' text into the template

\usepackage{setspace} % Line spacing

\usepackage[title]{appendix}

\usepackage{indentfirst}

\usepackage{helvet}

\usepackage{varioref}

\usepackage{hyperref}

\usepackage{cleveref}

\renewcommand{\familydefault}{\sfdefault}



% Margins

\setlength\parindent{24pt}

\topmargin=-0.45in

\evensidemargin=-.25in

\oddsidemargin=-.25in

\textwidth=7.0in

\textheight=9.0in

\headsep=0.25in 



\onehalfspacing

\newcommand{\quotes}[1]{``#1''}



% Set up the header and footer

\pagestyle{fancy}

\lhead{\firstxmark} % Top right header

\chead{\projecttitle} % Top left header

\rhead{\lastxmark} % Top right header

\lfoot{\firstxmark} % Bottom left footer

\rfoot{Page\ \thepage\ of\ \pageref{LastPage}} % Bottom right footer

\renewcommand\headrulewidth{0.4pt} % Size of the header rule

\renewcommand\footrulewidth{0.4pt} % Size of the footer rule



\setlength\parindent{0pt} % Removes all indentation from paragraphs

   

%----------------------------------------------------------------------------------------

%	NAME AND CLASS SECTION

%----------------------------------------------------------------------------------------



\newcommand{\projecttitle}{MDDS OS \\ User Guide} % Project Title

\newcommand{\hmwkAuthorName}{Yuchen Tian and Michael Tran} % Your name



%----------------------------------------------------------------------------------------

%	TITLE PAGE

%----------------------------------------------------------------------------------------



\title{
\vspace{2in}
\textmd{\textbf{\projecttitle} \\}
\vspace{5in}
}



\author{\textbf{\hmwkAuthorName}}

\date{} % Insert date here if you want it to appear below your name



%----------------------------------------------------------------------------------------



\begin{document}



\maketitle



%----------------------------------------------------------------------------------------

%	TABLE OF CONTENTS

%----------------------------------------------------------------------------------------



\newpage

\tableofcontents

\newpage



%----------------------------------------------------------------------------------------

%	Intro

%----------------------------------------------------------------------------------------



%----------------------------------------------------------------------------------------

%	Intro

%----------------------------------------------------------------------------------------


\begin{figure}[H]
	\centering
	\includegraphics[width=\linewidth]{Untitled}
\end{figure}


\section{Commands}

You may either use the buttons or type the commands into the input field on the bottom.

\subsection{Exe}

Argument(s): $n$ where $n$ is the number of cycles to simulate. You must enter $y/n$ in the console to continue the simulation.
If $n = 0$, the simulation will run until all processes are finished. The Exe button is $exe$ $0$ by default.

\subsection{Proc}

Argument(s): None. Displays all current processes.
This function is made obsolete because the information is constantly updated in the System Resource Monitor.

\subsection{Gen}

Argument(s): $n$ where $n$ is the number of process(es) to generate. The Gen button is $gen$ $5$ by default.

\subsection{Load}

Argument(s): $name$ where $name$ is the file name (without extension) of the program you wish to load.

Example: $load$ $word$ will load the word.txt file into the simulation.

\subsection{Mem}

Argument(s): None. Displays current RAM usage.
This function is made obsolete because the information is constantly updated in the System Resource Monitor.

\subsection{Reset}

Argument(s): None. Forces the simulation to stop and resets all components of the system.

\subsection{Clean}

Argument(s): None. Clears the console (left) screen.

\subsection{Exit}

Argument(s): None. Finally! You don't have to look at this thing anymore.

\section{Requirements}

This simulation is written and compiled in Java SDK 9.0.

GUI was partially designed with IntelliJ IDEA.

To manually load program files, you must place them in the same folder as MDS\_OS.jar.



\end{document}
